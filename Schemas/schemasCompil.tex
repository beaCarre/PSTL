\documentclass[a4paper, 11pt]{report}
\addtolength{\hoffset}{-1cm}
\addtolength{\textwidth}{2cm}
\usepackage[utf8]{inputenc}
\usepackage[frenchb]{babel}
\usepackage[T1]{fontenc}

\usepackage{multicol}
\usepackage{listings}

\usepackage{color}
\definecolor{lightgray}{rgb}{.9,.9,.9}
\definecolor{darkgray}{rgb}{.4,.4,.4}
\definecolor{purple}{rgb}{0.65, 0.12, 0.82}

\lstnewenvironment{OCaml}
                  {\lstset{
                      language=[Objective]Caml,
                      breaklines=true,
                      showstringspaces=false,
                      commentstyle=\color{purple},
                      stringstyle=\color{red},
                      identifierstyle=\ttfamily,
                      keywordstyle=\color{blue},
                      basicstyle=\footnotesize,
                      %xleftmargin=0.08\textwidth
                    }
                  }
                  {}
\newcommand{\class}{\ttfamily\textit{class}}
\newcommand{\interface}{\ttfamily\textit{interface}}
\newcommand{\name}{\ttfamily\textit{name}}
\newcommand{\package}{\ttfamily\textit{package}}
\newcommand{\ident}{\footnotesize\textbf{ident}}
  \lstdefinelanguage{idlgrammar}{
  morekeywords={package,abstract,extends,class,implements,static,final,<ini>,interface,callback,array,[,],{,},
    name ,void,boolean,byte,char,short,int,long,float,double,string},
  alsoletter=[]{},
}
 \lstnewenvironment{idl}
                  {\lstset{
                      language=idlgrammar,
                      breaklines=true,
                      showstringspaces=false,
                      keywordstyle=\ttfamily\color{blue},
                      identifierstyle=\ttfamily\textit,
                      basicstyle=\footnotesize,
                     escapeinside={(*}{*)},
                      %xleftmargin=0.08\textwidth
                    }
                  }
                  {} 
%%%%%%%%%%%%%%%%%%%%%%%%%%%%%%%%%%%%%%%%%%%%%%%%%%%%%%%%%%%%%%%%%%%%%%%%


\begin{document}
\chapter{}
Béatrice CARRE

\section*{Introduction}
La génération de code se fait en plusieurs passes :
\begin{itemize}
\item analyse lexicale et analyse syntaxique de l' idl donnant un
  ast de type Idl.file.
\item vérification des types de l'ast, donnant un nouvel ast de type
  CIdl.file.
\item la génération des fichiers stub java nécessaires pour un appel callback 
\item la génération à partir de l'ast CIdl.file du fichier .ml
\item la génération à partir du CIdl.file du fichier .mli
\end{itemize}
Ces différentes étapes seront présentées plus en profondeur.

\section{La syntaxe de l'idl}
La syntaxe du langage d'interface est donné en annexe, en utilisant la
notation BNF.

Les symboles < et > encadrent des règles optionnelles,
les terminaux sont en bleu, et les non-terminaux sont en italique.


\section{lexing parsing}
La première phase est celle d'analyse lexicale et syntaxique,
séparant l'idl en lexèmes et construisant l'AST, défini par Idl.file,
dont la structure : est définie en annexe

\section{check}
Vient ensuite la phase d'analyse sémantique, analysant l'AST obtenue par la
phase précédente, vérifiant si le programme est correct, et
construisant une liste de CIdl.clazz, restructurant chaque classe ou interface définie dans l'idl. 
Le module Cidl définit le nouvel AST allant être manipulé dans les passes de
génération de code. Il est décrit en annexe.

\section{génération stub\_file}
//TODO









%%%%%%%%%%%%%%%%%%%%%%%%%%%%   SCHEMAS   %%%%%%%%%%%%%%%%%%%%%%%%%%%%%%


\section{génération .ml}

La génération de ce code se fait en plusieurs passes sur l'ast obtenu
après ces précédents phases, le CIdl.file.

\subsection{schémas de compilation}
La génération de code rend une liste de valeurs imprimées (dans le
fichier engendré), en modifiant
Nous considérons l'environnement contenant les :
\begin{itemize}
\item package (:string) le nom du package courant
\item class\_env, un structure contenant les variables nécessaires à l
TODO

\item isInterface (:bool) est vrai si la déclaration courante est une
  interface. Faux si c'ets une classe.
\item isCallback (:bool) si la classe courante a l'attribut callback
\item isAbstractC (:bool) si la classe courante est abstraite
\item classname (:string) 
\item isAbstractElt (:bool) util?TODO
\item isStaticElt (:bool) util?TODO
\item isFinalElt (:bool) util?TODO
\end{itemize} 
Et les fonctions suivantes :
\begin{itemize}
\item init\_class\_env ()
\end{itemize} 
\noindent
$[\![ package\ <package>^* ]\!]$$\longrightarrow$

$[\![ package ]\!]$
\begin{OCaml}
   init_env ();
\end{OCaml} 

$[\![ package ]\!]$ 
\newline
\ 
\newline
\noindent
$[\![ decl\ <decl>^* ]\!]$$\longrightarrow$
% classes et interfaces sans package



\newline
\ 
\newline
\noindent
$[\![ package\ {\color{blue}qname}\ ;\ decl <decl>^* ]\!]$$\longrightarrow$
% metttre package name dans env



\newline
\ 
\newline
\noindent
$[\![ class\ {\color{blue}qname}\ ]\!]$$\longrightarrow$
% 1 type objet t 
% 1 classe encapsulante W de type t
% 1 a n classes (Ci), sous-classes de W (1 par constructeur)
% 1 fonction instanceof pr ce type
% 1 fonction de cast pour ce type

\noindent
$[\![{\color{blue}class}\  name\ {\color{blue}extends}\ qname ]\!]$$\longrightarrow$
%  

\noindent
$[\![{\color{blue}class}\  name\ {\color{blue}extends}\ qname ]\!]$$\longrightarrow$
% 

\noindent
$[\![{\color{blue}class}\  name\ {\color{blue}extends}\ qname ]\!]$$\longrightarrow$
% 


\section{génération .mli}
 












%%%%%%%%%%%%%%%%%%%%%%%%%%%%   ANNEXE   %%%%%%%%%%%%%%%%%%%%%%%%%%%%%%


\newpage
\section*{Annexe}


\subsection*{BNF}
\begin{idl}
(*\class*)

file ::= (*\package*) <(*\package*)>*
  	| decl <decl>*
 
(*\package*) ::= package qname ; decl <decl>*

decl ::= (*\class*)
  	|(*\interface*)
 
(*\class*) ::= <[attributes]> <abstract> class (*\name*)
  	  < extends qname >
  	  < implements qname <, qname>* >
  	  { <class_elt ;>* }
class_elt ::= <[ attributes ]> <static> <final> type (*\name*)
            | <[ attributes ]> <static> <abstract> type (*\name*) (<args>)
            | [ attributes ] <init> (<args>)
 
(*\interface*) ::= <[ attributes ]> interface (*\name*)
  	       < extends qname <, qname>* >
  	      { <interface_elt;>* }
interface_elt ::= 
     <[ attributes ]> type (*\name*)
   | <[ attributes ]> type (*\name*) (<args>)
 
args ::= arg <, arg>*
arg ::= <[ attributes ]> type <(*\name*)>
 
attributes ::= 	attribute <, attribute>*
attribute ::= name (*\ident*)
  	    | callback
  	    | array
 
type ::= basetype
       | object
       | basetype [ ]
basetype ::= void
           | boolean
           | byte
           | char
           | short
           | int
           | long
           | float
           | double
           | string
object := qname
qname ::= (*\name*)<.(*\name*)>*
(*\name*) ::= (*\ident*)
\end{idl}
\newpage
\subsection*{Module Idl}
\begin{multicols}{2}
\begin{OCaml}
(**  module Idl  *)

type ident = {
    id_location: Loc.t;
    id_desc: string 
  }
type qident = {
    qid_location: Loc.t;
    qid_package: string list;
    qid_name: ident;
  }
type type_desc = 
    Ivoid  
  | Iboolean
  | Ibyte
  | Ishort
  | Icamlint
  | Iint
  | Ilong
  | Ifloat
  | Idouble
  | Ichar
  | Istring
  | Itop
  | Iarray of typ
  | Iobject of qident
and typ = {
    t_location: Loc.t;
    t_desc: type_desc;
  }
type modifier_desc = 
  | Ifinal 
  | Istatic 
  | Iabstract
and modifier = {
    mo_location: Loc.t; 
    mo_desc: modifier_desc;
}
type ann_desc =
  | Iname of ident
  | Icallback
  | Icamlarray
and annotation = {
    an_location: Loc.t; 
    an_desc: ann_desc;
}
type arg = {
    arg_location: Loc.t; 
    arg_annot: annotation list;
    arg_type: typ
}   
type init = {
    i_location: Loc.t;
    i_annot: annotation list; 
    i_args: arg list;
}   
type field = {
    f_location: Loc.t;
    f_annot: annotation list; 
    f_modifiers: modifier list;
    f_name: ident;
    f_type: typ
}
type mmethod = { 
    m_location: Loc.t;
    m_annot: annotation list;
    m_modifiers: modifier list;
    m_name: ident;
    m_return_type: typ;
    m_args: arg list
}
type content = 
    | Method of mmethod 
    | Field of field
type def = {
    d_location: Loc.t;
    d_super: qident option;
    d_implements: qident list;
    d_annot: annotation list;
    d_interface: bool;
    d_modifiers: modifier list;
    d_name: ident;
    d_inits: init list;
    d_contents: content list;
}
type package = {
    p_name: string list;
    p_defs: def list;
}   
type file = package list
 
\end{OCaml}
\end{multicols}
\newpage
\subsection*{Module CIdl}
\begin{OCaml}
(**  module CIdl  *)
type typ =
  | Cvoid
  | Cboolean (** boolean -> bool *)
  | Cchar (** char -> char *)
  | Cbyte (** byte -> int *)
  | Cshort (** short -> int *)
  | Ccamlint (** int -> int<31> *)
  | Cint (** int -> int32 *)
  | Clong (** long -> int64 *)
  | Cfloat (** float -> float *)
  | Cdouble (** double -> float *)
  | Ccallback of Ident.clazz
  | Cobject of object_type (** object -> ... *)
and object_type = 
  | Cname of Ident.clazz (** ... -> object *)
  | Cstring (** ... -> string *)
  | Cjavaarray of typ (** ... -> t jArray *) 
  | Carray of typ (** ... -> t array *) 
  | Ctop

type clazz = {
    cc_abstract: bool;
    cc_callback: bool;
    cc_ident: Ident.clazz;
    cc_extend: clazz option; (* None = top *)
    cc_implements: clazz list;
    cc_all_inherited: clazz list; (* tout jusque top ... (et avec les interfaces) sauf elle-meme. *)
    cc_inits: init list;
    cc_methods: mmethod list; (* methodes + champs *)
    cc_public_methods: mmethod list; (* methodes declarees + celles heritees *)
    cc_static_methods: mmethod list; 
  }
and mmethod_desc = 
  | Cmethod of bool * typ * typ list (* abstract, rtype, args *)
  | Cget of typ
  | Cset of typ
and mmethod = {
    cm_class: Ident.clazz;
    cm_ident: Ident.mmethod; 
    cm_desc: mmethod_desc;
  }         
and init = {
    cmi_ident: Ident.mmethod;
    cmi_class: Ident.clazz;
    cmi_args: typ list;
  }
type file = clazz list
\end{OCaml}

\subsection*{module Ident}
\begin{OCaml}
(* module Ident  *)
(* le type des identifiants de classe de l'IDL *)
type clazz = {
    ic_id: int;
    ic_interface: bool;
    ic_java_package: string list;
    ic_java_name: string;
    ic_ml_name: string;
    ic_ml_name_location: Loc.t;
    ic_ml_name_kind: ml_kind;
  }
type mmethod = {
    im_java_name: string;
    im_ml_id: int; (** entier unique pour une nom ml *)
    im_ml_name: string;
    im_ml_name_location:Loc.t;
    im_ml_name_kind: ml_kind;
  }
\end{OCaml}
\ 

\subsection*{idl\_camlgen.make ast}
\ 
\newline
Type jni

\emph{MlClass.make\_jni\_type}
\newline
Class type

\emph{MlClass.make\_class\_type}
\newline
Cast JNI

\emph{MlClass.make\_jniupcast}

\emph{MlClass.make\_jnidowncast}
\newline
Fonction d'allocation

\emph{MlClass.make\_alloc}

\emph{MlClass.make\_alloc\_stub}
\newline
Capsule / souche

\emph{MlClass.make\_wrapper}
\newline
Downcast utilisateur

\emph{MlClass.make\_downcast}

\emph{MlClass.make\_instance\_of}
\newline
Tableaux

\emph{MlClass.make\_array}
\newline
Fonction d'initialisation

\emph{MlClass.make\_fun}
\newline
Classe de construction

\emph{MlClass.make\_class}
\newline
fonctions / methodes static

\emph{MlClass.make\_static}


\end{document}

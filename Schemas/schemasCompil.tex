\documentclass[a4paper, 11pt]{report}
\addtolength{\hoffset}{-1cm}
\addtolength{\textwidth}{2cm}
\usepackage[utf8]{inputenc}
\usepackage[frenchb]{babel}
\usepackage[T1]{fontenc}

\usepackage{multicol}
\usepackage{listings}

\usepackage{color}
\definecolor{lightgray}{rgb}{.9,.9,.9}
\definecolor{darkgray}{rgb}{.5,.2,.2}
\definecolor{purple}{rgb}{0.65, 0.12, 0.82}
\definecolor{brown}{RGB}{140, 0, 0}

\lstnewenvironment{OCaml}
                  {\lstset{
                      language=[Objective]Caml,
                      breaklines=true,
                      showstringspaces=false,
                      commentstyle=\color{red},
                      stringstyle=\color{darkgray},
                      identifierstyle=\ttfamily,
                      keywordstyle=\color{blue},
                      basicstyle=\footnotesize,
                      escapeinside={/*}{*/},
                      %xleftmargin=0.08\textwidth
                    }
                  }
                  {}
\newcommand{\class}{\ttfamily\textit{class}}
\newcommand{\interface}{\ttfamily\textit{interface}}
\newcommand{\name}{\ttfamily\textit{name}}
\newcommand{\package}{\ttfamily\textit{package}}
\newcommand{\ident}{\footnotesize\textbf{ident}}
\newcommand{\fun}[1]{\ttfamily\textbf{#1}}

  \lstdefinelanguage{idlgrammar}{
  morekeywords={package,abstract,extends,class,implements,static,final,<ini>,interface,callback,array,[,],{,},
    name ,void,boolean,byte,char,short,int,long,float,double,string},
  alsoletter=[]{},
}
 \lstnewenvironment{idl}
                  {\lstset{
                      language=idlgrammar,
                      breaklines=true,
                      showstringspaces=false,
                      keywordstyle=\ttfamily\color{blue},
                      identifierstyle=\ttfamily\textit,
                      basicstyle=\footnotesize,
                     escapeinside={(*}{*)},
                      %xleftmargin=0.08\textwidth
                    }
                  }
                  {} 
%%%%%%%%%%%%%%%%%%%%%%%%%%%%%%%%%%%%%%%%%%%%%%%%%%%%%%%%%%%%%%%%%%%%%%%%


\begin{document}
\chapter{}
Béatrice CARRE

\section*{Introduction}
La génération de code se fait en plusieurs passes :
\begin{itemize}
\item analyse lexicale et analyse syntaxique de l' idl donnant un
  ast de type Idl.file.
\item vérification des types de l'ast, donnant un nouvel ast de type
  CIdl.file.
\item la génération des fichiers stub java nécessaires pour un appel callback 
\item la génération à partir de l'ast CIdl.file du fichier .ml
\item la génération à partir du CIdl.file du fichier .mli
\end{itemize}
Ces différentes étapes seront présentées plus en profondeur.

\section{La syntaxe de l'idl}
La syntaxe du langage d'interface est donné en annexe, en utilisant la
notation BNF.

Les symboles < et > encadrent des règles optionnelles,
les terminaux sont en bleu, et les non-terminaux sont en italique.


\section{lexing parsing}
La première phase est celle d'analyse lexicale et syntaxique,
séparant l'idl en lexèmes et construisant l'AST, défini par Idl.file,
dont la structure : est définie en annexe

\section{check}
Vient ensuite la phase d'analyse sémantique, analysant l'AST obtenue par la
phase précédente, vérifiant si le programme est correct, et
construisant une liste de CIdl.clazz, restructurant chaque classe ou interface définie dans l'idl. 
Le module Cidl définit le nouvel AST allant être manipulé dans les passes de
génération de code. Il est décrit en annexe.

\section{génération stub\_file}
//TODO









%%%%%%%%%%%%%%%%%%%%%%%%%%%%   SCHEMAS   %%%%%%%%%%%%%%%%%%%%%%%%%%%%%%


\section{génération .ml}

La génération de ce code se fait en plusieurs passes sur l'ast obtenu
après ces précédents phases, le CIdl.file.

\subsection{schémas de compilation}
La génération de code rend une liste de valeurs imprimées (dans le
fichier engendré), en modifiant
Nous considérons un environnement contenant les  variables suivantes : 
\begin{OCaml}
let package := ""

let isInterface := false
let decl_name := ""
let isCallback := false
let isAbstractDecl := false
let extends := "JniHierarchy.top"
let implements := []
let init_list := {initname; arg*} list
\end{OCaml}
Et les fonctions suivantes :
\begin{OCaml}
let init_env () =
  package := "";
  init_class_env ()
let init_class_env () =
  isInterface := false;
  decl_name := "";
  isCallback := false;
  isAbstractDecl := false;
  extends := "";
  implements := []
\end{OCaml}
\begin{multicols}{2}
\textbf{file}
\newline
\noindent
$[\![ package^* ]\!]$$\longrightarrow$

$[\![ hd(package^*) ]\!]$
\begin{OCaml}
   init_env ();
\end{OCaml} 

$[\![ tl(package^*) ]\!]$ 
\newline
\ 
\newline
\noindent
$[\![ decl^* ]\!]$$\longrightarrow$
% classes et interfaces sans package

$[\![ hd(decl^*) ]\!]$
\begin{OCaml}
   init_class_env ();
\end{OCaml} 

$[\![ tl(decl^*) ]\!]$ 
\newline
\ 
\newline
\end{multicols}
\textbf{package}
\newline
\noindent
$[\![ package\ {\color{blue}qname}\ ;\ decl^* ]\!]$$\longrightarrow$
% metttre package name dans env

$[\![ decl^* ]\!]_package:=qname$ 

\ 
\newline

\textbf{class}
\newline
\noindent
$[\![ {\color{blue}class}\ ]\!]_{isInterface, package, extends, implements, init\_list:\{initname, args^*\}^*, }$$\longrightarrow$
% 1 type objet t 
% 1 classe encapsulante W de type t
% 1 a n classes (Ci), sous-classes de W (1 par constructeur)
% 1 fonction instanceof pr ce type
% 1 fonction de cast pour ce type
\begin{OCaml}
(** type jni.obj t *)
"type _jni_j"^name^" = Jni.obj;;"
(** classe encapsulante *)
"class type j"^name^" ="
(** herite top car extends = "" *)
   "object inherit "^extends^""
    make_implements implements
   "method _get_jni_j"^name^" : _jni_j"^name^"
   end;;"
(** upcast jni *)
"let __jni_obj_of_jni_j"^name^" (java_obj : _jni_j"^name^") =
   (Obj.magic : _jni_j"^name^" -> Jni.obj) java_obj;;"
(** downcast jni *)
"let __jni_j"^name^"_of_jni_obj =
   let clazz =
      try Jni.find_class \""^package^"/"^name^"\"
   with | _ -> failwith "Class not found : "^package^"."^name^"."
   in
      fun (java_obj : Jni.obj) ->
        if not (Jni.is_instance_of java_obj clazz)
        then failwith \"cast error\" : j"^name^" ("^package^"/"^name^")
        else (Obj.magic java_obj : _jni_j"^name^");;"
(* allocation *)
/* \fun{make\_alloc} */package name isInterface;
(* capsule *)
/* \fun{make\_capsule} */package name extends implements;
(* downcast utilisateur *)
"let j"^name^"_of_top (o : JniHierarchy.top) : j"^name^" =
    new _capsule_j"^name^" (__jni_j"^name^"_of_jni_obj o#_get_jniobj);;"
(* instance_of *)
"let _instance_of_j"^name^" =
   let clazz = Jni.find_class \""^package^"/"^name^"\"
    in fun (o : JniHierarchy.top) -> Jni.is_instance_of o#_get_jniobj clazz;;"
(* tableaux *)
"let _new_jArray_j"^name^" size =
    let java_obj = Jni.new_object_array size (Jni.find_class \""^package^"/"^name^"\")
    in
      new JniArray._Array Jni.get_object_array_element Jni.
        set_object_array_element (fun jniobj -> new _capsule_j"^name^" jniobj)
        (fun obj -> obj#_get_jni_j"^name^") java_obj;;"
"let jArray_init_j"^name^" size f =
    let a = _new_jArray_j"^name^" size
    in (for i = 0 to pred size do a#set i (f i) done; a);;"
make_inits init_list;
\end{OCaml}

\newpage
\
\newline
\textbf{ interface }
\newline
\noindent
$[\![{\color{blue}interface}\  name ]\!]$$\longrightarrow$
% 

$[\![{\color{blue}class}\  name ]\!]_{isInterface:=true}$
\newline
\ 
\newline
\noindent
$[\![{\color{blue}interface}\  name\ {\color{blue}implements}\ qname* ]\!]$$\longrightarrow$
% 

\begin{OCaml}
let make_alloc name isInterface =
   if not isInterface then
      "let _alloc_j"^name^" =
       let clazz = Jni.find_class "^package^"/"^name^"
       in fun () -> (Jni.alloc_object clazz : _jni_j"^name^");;"
\end{OCaml}

\ 
\newline
\textbf{extends, implements}
\newline
\noindent
$[\![{\color{blue}class}\  name\ {\color{blue}implements}\ qname* ]\!]$$\longrightarrow$
% 

let l = String.split "," qname* in

$[\![{\color{blue}class}\  name ]\!]_{implements:=l}$
\newline
\ 
\newline
\noindent
\noindent
$[\![  {\color{blue}class}\ name\ {\color{blue}extends} qname ]\!]$$\longrightarrow$
% verification extends

$[\![  {\color{blue}class}\ name]\!]_{extends:=qname}$

\noindent
% capsule selon extends extends
\begin{OCaml}
let make_inherits_implements implements =
   List.iter (fun int -> "inherit j"^int ) implements

let make_capsule package name extends implements =
     "class _capsule_j"^name^" =
       let clazz = Jni.find_class \""^package^"/"^name^"\"
       in"
    (* si la class extends une autre *)
    if extends != "JniHierarchy.top" then ( 
        " let _ =
           if not (Jni.is_assignable_from clazz (Jni.find_class \""^package^"/"^extends^"\"))
           then
           failwith \"Wrong super class in IDL : "^package^"/"^classname^" not extends"^package^"/"^extends^".\"
           else ()
         in "
    );
   (* pour chaque interface implementee : *)
    List.iter (fun int ->
        "let _ =
           if not (Jni.is_assignable_from clazz (Jni.find_class "^package^"/"^int^"))
           then
             failwith \"Wrong implemented interface in IDL : "^package^"."^name^ "does not implements "^package^"."^int^".\"
           else ()
        in"
    ) implements;
      (* fonction commune *)
       "fun (jni_ref : _jni_j"^name^") ->
         let _ =
           if Jni.is_null jni_ref
           then raise (JniHierarchy.Null_object \""^package^"/"^name^"\")
           else ()
         in
         (* objet avec les methodes d'acces a toutes les classes/interfaces implementee ou heritees *)
         object (self)
           method _get_jni_j"^name^" = jni_ref"
      if extends != "JniHierarchy.top" then (
          "  method _get_jni_j"^extends^" = jni_ref"
      );
      List.iter (fun int -> "method _get_jni_j"^int^" = jni_ref;;") implements;
          " inherit JniHierarchy.top jni_ref
         end;;"
  
\end{OCaml}



\ 
\newline
\noindent
\textbf{ inits }
\newline
\noindent
$[\![... {\color{blue}class}\  name\ ... \{ ...
    $[$ {\color{blue}name}\ initname $]${\color{blue} <init>} (arg*); ...\} ]\!]$$\longrightarrow$
% 

$[\![{\color{blue}class}\  name ]\!]_{init\_list:=\{initname;arg*\}::!init\_list}$

\begin{OCaml}

let getArgsString args =
String.concat (List.map (fun arg -> 
   match arg with
   | ("int",name,attr) -> "I"
   | ("boolean",name,attr) -> "Z"
   | ("string",name,attr) -> "LJava/lang/String"
     ...
   | (obj,name,attr) -> "L"^package^obj^";"
in

(* todo : fun pr recup jtype, method get_jni_jtype et jni.mltype *)

let makefun = (* exemples : *)
 "fun (java_obj : _jni_j"^classname^") _p0 _p1 ->
      let _p1 = _p1 in
      let _p0 = _p0
      in
        Jni.call_nonvirtual_void_method java_obj clazz id
          [| Jni. _p0; Jni.Camlint _p1 "|];;" 
 fun (java_obj : _jni_jOu) (_p0 : jFormule) (_p1 : jFormule) ->
      let _p1 = _p1#_get_jni_jFormule in
      let _p0 = _p0#_get_jni_jFormule
      in
        Jni.call_nonvirtual_void_method java_obj clazz id
          [| Jni.Obj _p0; Jni.Obj _p1 |];;
in"

"let _init_"^initname^" =
  let clazz = Jni.find_class \""^package^"/"^classname^"\" in
  let id =
    try Jni.get_methodID clazz \"<init>\" \"("^(getArgsString args)^")V\"
    with
    | _ ->
        failwith
          \"Unknown constructor from IDL in class \""^package^"."^classname^"\" : \""^classname^"(int,int)\".\"
  in
    fun (java_obj : _jni_j"^classname^") _p0 _p1 ->
      let _p1 = _p1 in
      let _p0 = _p0
      in
        Jni.call_nonvirtual_void_method java_obj clazz id
          [| Jni."^^" _p0; Jni.Camlint _p1 |];;
let _init_"^initname^" =
  let clazz = Jni.find_class \""^package^"/"^classname^"\" in
  let id =
    try Jni.get_methodID clazz \"<init>\" \"()V\"
    with
    | _ ->
        failwith
          \"Unknown constructor from IDL in class \""^package^"."^classname^"\" : \""^classname^"()\".\"
  in
    fun (java_obj : _jni_j"^classname^") ->
      Jni.call_nonvirtual_void_method java_obj clazz id [| 

       |];;"

\end{OCaml}
\ 
\newline
\noindent
\textbf{ attributs }
\newline
\noindent
\ 
\newline
\noindent
\textbf{ methodes }
\newline
\noindent



\section{génération .mli}
 












%%%%%%%%%%%%%%%%%%%%%%%%%%%%   ANNEXE   %%%%%%%%%%%%%%%%%%%%%%%%%%%%%%


\newpage
\section*{Annexe}


\subsection*{BNF}
\begin{idl}
(*\class*)

file ::= (*\package*) <(*\package*)>*
  	| decl <decl>*
 
(*\package*) ::= package qname ; decl <decl>*

decl ::= (*\class*)
  	|(*\interface*)
 
(*\class*) ::= <[attributes]> <abstract> class (*\name*)
  	  < extends qname >
  	  < implements qname <, qname>* >
  	  { <class_elt ;>* }
class_elt ::= <[ attributes ]> <static> <final> type (*\name*)
            | <[ attributes ]> <static> <abstract> type (*\name*) (<args>)
            | [ attributes ] <init> (<args>)
 
(*\interface*) ::= <[ attributes ]> interface (*\name*)
  	       < extends qname <, qname>* >
  	      { <interface_elt;>* }
interface_elt ::= 
     <[ attributes ]> type (*\name*)
   | <[ attributes ]> type (*\name*) (<args>)
 
args ::= arg <, arg>*
arg ::= <[ attributes ]> type <(*\name*)>
 
attributes ::= 	attribute <, attribute>*
attribute ::= name (*\ident*)
  	    | callback
  	    | array
 
type ::= basetype
       | object
       | basetype [ ]
basetype ::= void
           | boolean
           | byte
           | char
           | short
           | int
           | long
           | float
           | double
           | string
object := qname
qname ::= (*\name*)<.(*\name*)>*
(*\name*) ::= (*\ident*)
\end{idl}
\newpage
\subsection*{Module Idl}
\begin{multicols}{2}
\begin{OCaml}
(**  module Idl  *)

type ident = {
    id_location: Loc.t;
    id_desc: string 
  }
type qident = {
    qid_location: Loc.t;
    qid_package: string list;
    qid_name: ident;
  }
type type_desc = 
    Ivoid  
  | Iboolean
  | Ibyte
  | Ishort
  | Icamlint
  | Iint
  | Ilong
  | Ifloat
  | Idouble
  | Ichar
  | Istring
  | Itop
  | Iarray of typ
  | Iobject of qident
and typ = {
    t_location: Loc.t;
    t_desc: type_desc;
  }
type modifier_desc = 
  | Ifinal 
  | Istatic 
  | Iabstract
and modifier = {
    mo_location: Loc.t; 
    mo_desc: modifier_desc;
}
type ann_desc =
  | Iname of ident
  | Icallback
  | Icamlarray
and annotation = {
    an_location: Loc.t; 
    an_desc: ann_desc;
}
type arg = {
    arg_location: Loc.t; 
    arg_annot: annotation list;
    arg_type: typ
}   
type init = {
    i_location: Loc.t;
    i_annot: annotation list; 
    i_args: arg list;
}   
type field = {
    f_location: Loc.t;
    f_annot: annotation list; 
    f_modifiers: modifier list;
    f_name: ident;
    f_type: typ
}
type mmethod = { 
    m_location: Loc.t;
    m_annot: annotation list;
    m_modifiers: modifier list;
    m_name: ident;
    m_return_type: typ;
    m_args: arg list
}
type content = 
    | Method of mmethod 
    | Field of field
type def = {
    d_location: Loc.t;
    d_super: qident option;
    d_implements: qident list;
    d_annot: annotation list;
    d_interface: bool;
    d_modifiers: modifier list;
    d_name: ident;
    d_inits: init list;
    d_contents: content list;
}
type package = {
    p_name: string list;
    p_defs: def list;
}   
type file = package list
 
\end{OCaml}
\end{multicols}
\newpage
\subsection*{Module CIdl}
\begin{OCaml}
(**  module CIdl  *)
type typ =
  | Cvoid
  | Cboolean (** boolean -> bool *)
  | Cchar (** char -> char *)
  | Cbyte (** byte -> int *)
  | Cshort (** short -> int *)
  | Ccamlint (** int -> int<31> *)
  | Cint (** int -> int32 *)
  | Clong (** long -> int64 *)
  | Cfloat (** float -> float *)
  | Cdouble (** double -> float *)
  | Ccallback of Ident.clazz
  | Cobject of object_type (** object -> ... *)
and object_type = 
  | Cname of Ident.clazz (** ... -> object *)
  | Cstring (** ... -> string *)
  | Cjavaarray of typ (** ... -> t jArray *) 
  | Carray of typ (** ... -> t array *) 
  | Ctop

type clazz = {
    cc_abstract: bool;
    cc_callback: bool;
    cc_ident: Ident.clazz;
    cc_extend: clazz option; (* None = top *)
    cc_implements: clazz list;
    cc_all_inherited: clazz list; (* tout jusque top ... (et avec les interfaces) sauf elle-meme. *)
    cc_inits: init list;
    cc_methods: mmethod list; (* methodes + champs *)
    cc_public_methods: mmethod list; (* methodes declarees + celles heritees *)
    cc_static_methods: mmethod list; 
  }
and mmethod_desc = 
  | Cmethod of bool * typ * typ list (* abstract, rtype, args *)
  | Cget of typ
  | Cset of typ
and mmethod = {
    cm_class: Ident.clazz;
    cm_ident: Ident.mmethod; 
    cm_desc: mmethod_desc;
  }         
and init = {
    cmi_ident: Ident.mmethod;
    cmi_class: Ident.clazz;
    cmi_args: typ list;
  }
type file = clazz list
\end{OCaml}

\subsection*{module Ident}
\begin{OCaml}
(* module Ident  *)
(* le type des identifiants de classe de l'IDL *)
type clazz = {
    ic_id: int;
    ic_interface: bool;
    ic_java_package: string list;
    ic_java_name: string;
    ic_ml_name: string;
    ic_ml_name_location: Loc.t;
    ic_ml_name_kind: ml_kind;
  }
type mmethod = {
    im_java_name: string;
    im_ml_id: int; (** entier unique pour une nom ml *)
    im_ml_name: string;
    im_ml_name_location:Loc.t;
    im_ml_name_kind: ml_kind;
  }
\end{OCaml}
\ 

\subsection*{idl\_camlgen.make ast}
\ 
\newline
Type jni

\emph{MlClass.make\_jni\_type}
\newline
Class type

\emph{MlClass.make\_class\_type}
\newline
Cast JNI

\emph{MlClass.make\_jniupcast}

\emph{MlClass.make\_jnidowncast}
\newline
Fonction d'allocation

\emph{MlClass.make\_alloc}

\emph{MlClass.make\_alloc\_stub}
\newline
Capsule / souche

\emph{MlClass.make\_wrapper}
\newline
Downcast utilisateur

\emph{MlClass.make\_downcast}

\emph{MlClass.make\_instance\_of}
\newline
Tableaux

\emph{MlClass.make\_array}
\newline
Fonction d'initialisation

\emph{MlClass.make\_fun}
\newline
Classe de construction

\emph{MlClass.make\_class}
\newline
fonctions / methodes static

\emph{MlClass.make\_static}


\end{document}

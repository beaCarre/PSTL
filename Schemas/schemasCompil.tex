\documentclass[a4paper, 11pt]{report}
\addtolength{\hoffset}{-1cm}
\addtolength{\textwidth}{2cm}
\usepackage[utf8]{inputenc}
\usepackage[frenchb]{babel}
\usepackage[T1]{fontenc}

\usepackage{multicol}
\usepackage{listings}

\usepackage{hyperref}

\usepackage{color}
\definecolor{lightgray}{rgb}{.9,.9,.9}
\definecolor{darkgray}{rgb}{.5,.2,.2}
\definecolor{purple}{rgb}{0.65, 0.12, 0.82}
\definecolor{brown}{RGB}{140, 0, 0}

\lstnewenvironment{OCaml}
                  {\lstset{
                      language=[Objective]Caml,
                      breaklines=true,
                      showstringspaces=false,
                      commentstyle=\color{red},
                      stringstyle=\color{darkgray},
                      identifierstyle=\ttfamily,
                      keywordstyle=\color{blue},
                      basicstyle=\footnotesize,
                      escapeinside={/*}{*/},
                      %xleftmargin=0.08\textwidth
                    }
                  }
                  {}
\lstnewenvironment{OCamlEx}
                  {\lstset{
                      language=[Objective]Caml,
                      breaklines=true,
                      showstringspaces=false,
                      commentstyle=\color{red},
                      stringstyle=\color{darkgray},
                      identifierstyle=\ttfamily,
                      keywordstyle=\color{blue},
                      basicstyle=\footnotesize,
                      escapeinside={/*}{*/},
                      frame=single,
                      numbers=left,
                      %xleftmargin=0.08\textwidth
                    }
                  }
                  {}
\newcommand{\class}{\ttfamily\textit{class}}
\newcommand{\interface}{\ttfamily\textit{interface}}
\newcommand{\name}{\ttfamily\textit{name}}
\newcommand{\package}{\ttfamily\textit{package}}
\newcommand{\ident}{\footnotesize\textbf{ident}}
\newcommand{\fun}[1]{\ttfamily\textbf{#1}}

\lstdefinelanguage{idlgrammar}{
  morekeywords={package,abstract,extends,class,implements,static,final,<ini>,interface,callback,array,[,],{,},
    name ,void,boolean,byte,char,short,int,long,float,double,string},
  alsoletter=[]{},
}
\lstnewenvironment{idl}
                  {\lstset{
                      language=idlgrammar,
                      breaklines=true,
                      showstringspaces=false,
                      keywordstyle=\ttfamily\color{blue},
                      identifierstyle=\ttfamily\textit,
                      basicstyle=\footnotesize,
                     escapeinside={(*}{*)},
                      %xleftmargin=0.08\textwidth
                    }
                  }
                  {} 
%%%%%%%%%%%%%%%%%%%%%%%%%%%%%%%%%%%%%%%%%%%%%%%%%%%%%%%%%%%%%%%%%%%%%%%%


\begin{document}
\chapter{}
Béatrice CARRE

\section*{Introduction}
La génération de code se fait en plusieurs passes :
\begin{itemize}
\item analyse lexicale et analyse syntaxique de l' idl donnant un
  ast de type Idl.file.
\item vérification des types de l'ast, donnant un nouvel ast de type
  CIdl.file.
\item la génération des fichiers stub java nécessaires pour un appel callback 
\item la génération à partir de l'ast CIdl.file du fichier .ml
\item la génération à partir du CIdl.file du fichier .mli
\end{itemize}
Ces différentes étapes seront présentées plus en profondeur.

\section{La syntaxe de l'idl}
La syntaxe du langage d'interface est donné en annexe, en utilisant la
notation BNF.

Les symboles < et > encadrent des règles optionnelles,
les terminaux sont en bleu, et les non-terminaux sont en italique.


\section{lexing parsing}
La première phase est celle d'analyse lexicale et syntaxique,
séparant l'idl en lexèmes et construisant l'AST, défini par Idl.file,
dont la structure : est définie en annexe

\section{check}
Vient ensuite la phase d'analyse sémantique, analysant l'AST obtenue par la
phase précédente, vérifiant si le programme est correct, et
construisant une liste de CIdl.clazz, restructurant chaque classe ou interface définie dans l'idl. 
Le module Cidl définit le nouvel AST allant être manipulé dans les passes de
génération de code. Il est décrit en annexe.

\section{génération stub\_file}
//TODO









%%%%%%%%%%%%%%%%%%%%%%%%%%%%   SCHEMAS   %%%%%%%%%%%%%%%%%%%%%%%%%%%%%%


\section{génération .ml}

La génération de ce code se fait en plusieurs passes sur l'ast obtenu
après ces précédents phases, le CIdl.file.

\subsection{schémas de compilation}
La génération de code rend du code OCaml (écrit dans un fichier).

\noindent
Nous considérons un environnement contenant les  variables suivantes, initialisées à leur valuer par défaut: 

$\rho$ = "" : le nom du \textbf{package} où trouver les classes définies.

$\Gamma$ = false : si la déclaration est une \textbf{interface}.

$\theta$ = false : si l'élément porte l'attribut \textbf{callback}.

$\alpha$ = false : si l'élément est déclaré \textbf{abstract}.

$\delta$ = "JniHierarchy.top" : la classe dont \textbf{extends} la classe courante.

$\Delta$ = [] : les interfaces qu'\textbf{implements} la classe courante.
\
\newline
\noindent
Et les fonctions suivantes :

init\_env () : réinitialise toutes les variables d'environnement 

init\_class\_env () : réinitialise toutes les variables d'environnement sauf $\rho$.

hd(elt*) : rend le premier élément de la list elt*.

tl(elt*) : rend la liste elt* privée de son premier élément.

\begin{multicols}{2}
\textbf{file}
\newline
\noindent
$[\![ package^* ]\!]$$\longrightarrow$

$[\![ hd(package^*) ]\!]$
\begin{OCaml}
   init_env ();
\end{OCaml} 

$[\![ tl(package^*) ]\!]$ 
\newline
\ 
\newline
\noindent
$[\![ decl^* ]\!]$$\longrightarrow$
% classes et interfaces sans package

$[\![ hd(decl^*) ]\!]$
\begin{OCaml}
   init_class_env ();
\end{OCaml} 

$[\![ tl(decl^*) ]\!]$ 
\newline
\ 
\newline

\textbf{package}
\newline
\noindent
$[\![ package\ {\color{blue}qname}\ ;\ decl^* ]\!]$$\longrightarrow$
% metttre package name dans env

$[\![ decl^* ]\!]_{\rho=qname}$ 

\ 
\newline

\textbf{ decl interface }
\newline
\noindent
$[\![{\color{blue}interface}\  name ]\!]$$\longrightarrow$

$[\![{\color{blue}class}\  name ]\!]_{\rho,\Gamma=true}$

\ 
\newline

\textbf{ decl class }
\newline
\noindent
$[\![$[$ {\color{blue}callabck}$]${\color{blue} class}\  name ]\!]_{\rho,\Gamma,\theta,\alpha}$$\longrightarrow$

$[\![{\color{blue}class}\  name ]\!]_{\rho,\Gamma,\theta=true,\alpha}$
\end{multicols}

\ 
\newpage
\noindent

\textbf{ decl class}
\newline
\noindent
$[\![ {\color{blue}class}\ CLASS\ 
 {\color{blue}extends}\  E \ 
 {\color{blue}implements}\  I1,I2... \{$

 $ \ \ \ attr1; attr2; ...;$

  $\ \ \ m1; m2; ...;$

  $\ \ \ init1; init2; ...;$

 $\} ]\!]_{\rho,CB}\longrightarrow$
% 1 type objet t 
% 1 classe encapsulante W de type t
% 1 a n classes (Ci), sous-classes de W (1 par constructeur)
% 1 fonction instanceof pr ce type
% 1 fonction de cast pour ce type
\ 
\newline

\begin{OCaml}
let clazz = Jni.find_class PACK/CLASS

(** type jni.obj t *)
"type _jni_jCLASS = Jni.obj"

(** classe encapsulante *)
"class type jCLASS =
   object inherit E
   inherits jI1
   inherits jI2 ...
   method _get_jni_jCLASS : _jni_jCLASS
   end"

(** upcast jni *)
"let __jni_obj_of_jni_jCLASS (java_obj : _jni_jCLASS) =
   (Obj.magic : _jni_jCLASS -> Jni.obj) java_obj"
(** downcast jni *)
"let __jni_jCLASS_of_jni_obj =
   fun (java_obj : Jni.obj) ->
     Jni.is_instance_of java_obj clazz"
 
(* allocation : si ce n'est pas une interface *)
"let _alloc_jCLASS =
     fun () -> (Jni.alloc_object clazz : _jni_jCLASS)"

(* capsule wrapper *)
"class _capsule_jCLASS = fun (jni_ref : _jni_jCLASS) ->
    object (self)
      method _get_jni_jCLASS = jni_ref
      method _get_jni_jE = jni_ref
      method _get_jni_jI1 = jni_ref
      method _get_jni_jI2 = jni_ref
      inherit JniHierarchy.top jni_ref
    end"

(* downcast utilisateur *)
"let jCLASS_of_top (o : TOP) : jCLASS =
    new _capsule_jCLASS (__jni_jCLASS_of_jni_obj o#_get_jniobj)"
(* instance_of *)
"let _instance_of_jCLASS =
    in fun (o : TOP) -> Jni.is_instance_of o#_get_jniobj clazz"

(* tableaux *)
"let _new_jArray_jCLASS size =
    let java_obj = Jni.new_object_array size (Jni.find_class \"PACK/CLASS\")
    in
      new JniArray._Array Jni.get_object_array_element Jni.
        set_object_array_element (fun jniobj -> new _capsule_jCLASS jniobj)
        (fun obj -> obj#_get_jni_jCLASS) java_obj"
"let jArray_init_jCLASS size f =
    let a = _new_jArray_jCLASS size
    in (for i = 0 to pred size do a#set i (f i) done; a)"

(* inits *)
\end{OCaml}

$[\![$[$ {\color{blue}name}\ init1 $]${\color{blue} <init>} (arg*); ...\}
]\!]$

$[\![$[$ {\color{blue}name}\ init2 $]${\color{blue} <init>} (arg*); ...\}
]\!]$

...
\begin{OCaml}
(* fonctions et  methodes statiques*)
 
    (*TODO*)

\end{OCaml}


\
\ 
\newline
\ 
\newline
Ce tableau représente le résultat des fonctions str, jni\_type, getJni, cast sur les types lors des générations des constructeurs ou des méthodes.

\noindent
\begin{tabular}{|l|c|c|c|c|}
  \hline
  TYPE & str & jni\_type & getJni & cast \\
  \hline
  void & " "||V & & & \\
  boolean & Z & Jni.Boolean \_pi & \_pi  & \_pi \\
  byte & B & Jni.Byte \_pi & \_pi & \_pi \\
  char & C & Jni.Char \_pi & \_pi & \_pi \\
  short & S & Jni.Short \_pi & \_pi & \_pi  \\
  int & I & Jni.Camlint \_pi & \_pi &  \_pi \\
  long & J & Jni.Long \_pi &\_pi  & \_pi \\
  float & F & Jni.Float \_pi & \_pi & \_pi \\
  double & D & Jni.Double \_pi & \_pi & \_pi \\
  string &LJava/lang/String;& Jni.Obj \_pi & Jni.string\_to\_java \_pi & \_pi \\
  pack/Obj& Lpack/Obj;& Jni.Obj \_pi & \_pi\#\_get\_jni\_jname & (\_pi : jObj) \\
  \hline
\end{tabular}
\ 
\newline
\noindent
\textbf{ inits }
\newline
\noindent
$[\![$[$ {\color{blue}name}\ INIT $]${\color{blue} <init>} (A0,
    A1, ...)]\!]_{}$$\longrightarrow$
% 

\begin{OCaml}
"let _init_INIT =
  let id = Jni.get_methodID clazz \"<init>\" 
            \"("(toStr A0)(toStr A1)...")V\"
  in
    fun (java_obj : _jni_jCLASS) "(cast A0) (cast A1) ..." -> 
      ...
      let _p1 = "(getJni A1)" in
      let _p0 = "(getJni A0)" in
      Jni.call_nonvirtual_void_method java_obj clazz id 
          [| "(jni\_type A0)"; "(jni\_type A1)"; ... |]

class INIT _p0 _p1 ... =
  let java_obj = _alloc_jCLASS ()
  in let _ = _init_INIT java_obj _p0 _p1 ...
    in object (self) inherit _capsule_jCLASS java_obj 
end"

\end{OCaml}
\ 
\newline
\noindent
\textbf{ attributs }
\newline
\noindent
\ 
$[\![ TYPE\ ATTR; ]\!]_{}$$\longrightarrow$

\begin{OCaml}
...
(* type class *)
"class type jCLASS =
  ...
   method set_ATTR : (j)TYPE -> unit
   method get_ATTR : unit -> (j)TYPE
   ... "
(* capsule *)
"class _capsule_jCLASS =
   let __fid_ATTR = try Jni.get_fieldID clazz \"ATTR\" "(toStr TYPE)" in
   ...
   fun (jni_ref : _jni_jCLASS) -> 
     object (self)
        method set_ATTR =
           fun "(castArg TYPE)" ->
              let _p = "(getJni TYPE)"
              in Jni.set_object_field jni_ref __fid_ATTR _p
        method get_ATTR =
        fun () ->
           (new _capsule_jCLASS (Jni.get_object_field jni_ref __fid_ATTR) :
           jCLASS)
        ...
   "
\end{OCaml}

\noindent
\textbf{ methodes }

\noindent
\begin{tabular}{|l|c|c|c|c|}
  \hline
  TYPE & str & jni\_type & getJni & cast \\
  \hline
  void & " "||V & & & \\
  boolean & Z & Jni.Boolean \_pi & \_pi  & \_pi \\
  byte & B & Jni.Byte \_pi & \_pi & \_pi \\
  char & C & Jni.Char \_pi & \_pi & \_pi \\
  short & S & Jni.Short \_pi & \_pi & \_pi  \\
  int & I & Jni.Camlint \_pi & \_pi &  \_pi \\
  long & J & Jni.Long \_pi &\_pi  & \_pi \\
  float & F & Jni.Float \_pi & \_pi & \_pi \\
  double & D & Jni.Double \_pi & \_pi & \_pi \\
  string &LJava/lang/String;& Jni.Obj \_pi & Jni.string\_to\_java \_pi & \_pi \\
  pack/Obj& Lpack/Obj;& Jni.Obj \_pi & \_pi\#\_get\_jni\_jname & (\_pi : jObj) \\
  \hline
\end{tabular}
\ 
\newline
\noindent
$[\![ TYPE METH (ARG1, ARG2, ...)]\!]_{}$$\longrightarrow$

\begin{OCaml}
...
(* type class *)
"class type jCLASS =
   ...
   method METH : ARG1 -> ARG2 -> ... -> TYPE
   ... "
(* capsule *)
"class _capsule_jCLASS =
   let __mid_METH = Jni.get_methodID clazz "mETH"
         \"("(toStr ARG1)(toStr ARG2)...")"(toStr TYPE)"\"
   ...
   in
   object (self)
"(*TODO*)"      (*method METHObj1Obj2 =
         fun (_p0 : jObj1) ->
           let _p0 = _p0#_get_jni_jObj1
           ...
             in
             (new _capsule_jObj2
               (Jni.call_"Object"_method jni_ref __mid_mETHObj1Obj2
               [| Jni.Obj _p0 |]) : jObj2)
      *)
      method METH =
         fun "(cast A0) (cast A1) ..." ->
           let _p2 = "(getJni ARG2)" in
           let _p1 = "(getJni ARG1)" in
           let _p0 = "(getJni ARG0)"
           in
             Jni.call_"(aJniType TYPE)"_method jni_ref __mid_METH
               [| "(jni\_type ARG0)"; "(jni\_type ARG1)"; ... |]
\end{OCaml}
//TODO : 
retour Obj dans methode
array
callback
\section{génération .mli}
 







\newpage
%%%%%%%%%%%%%%%%%%%%%%%%%%%%   OCaml-Java   %%%%%%%%%%%%%%%%%%%%%%%%%%%%%%

\section{Ocaml-Java}


%%%%%%%%%%%%%%%%%%%%%%%%%%%%   OCava   %%%%%%%%%%%%%%%%%%%%%%%%%%%%%%
\subsection{Génération de code pour Ocaml-Java}
L'idée est de partir de 
Le type top manipulé sera le type d'instance objet de Ocaml-Java :
\begin{OCamlEx}
type top = java'lang'Object java_instance;;
\end{OCamlEx}

Exception :
\begin{OCamlEx}
exception Null_object of string
\end{OCamlEx}

\noindent
Description des types manipulés par OCaml-Java.

\begin{tabular}{|l|l|}
  \hline
  type Java & description et exemple \\
  \hline
  java\_constructor & signature d'un constructeur  \\
  &  "java.lang.Object()" \\
  \hline
  java\_method & signature d'une méthode \\
  & "java.lang.String.lastIndexOf(string):int"\\
  \hline
  java\_field\_get & signature d'un attribut\\
  & "mypack.Point.x:int" \\
  \hline
  java\_field\_set & signature d'un attribut\\
  & "mypack.Point.x:int" \\
  \hline
  java\_type & classe, interface ou type Array\\
  & "java.lang.String"\\
  \hline
  java\_proxy & type d'une interface\\
  & "java.lang.Comparable"\\ 
  \hline
\end{tabular}

\noindent
Description du module Java

\begin{OCamlEx}
make : 'a java_constructor -> 'a
call : 'a java_method -> 'a
get : 'a java_field_get -> 'a
set : 'a java_field_set -> 'a
instanceof : 'a java_type -> 'b java_instance -> bool
cast : 'a java_type -> 'b java_instance -> 'a
proxy : 'a java_proxy -> 'a
\end{OCamlEx}


\ 
\newline
\textbf{class}

Le schéma de compilation de base pour une classe est largement allégé. 
En effet,la fonction downcast jni est inutile, puisqu'on a la fonction Java.cast, effectuant tout le travail.

De même, l'upcast -> est direct ? pour quoi ?

TODO : voir http://www.pps.univ-paris-diderot.fr/~henry/ojacare/doc/ojacare006.html. (** cast JNI, exporté pour préparer la fonction 'import' *)

L'allocation n'est pas non plus nécessaire, OCaml-Java gérant tout ça côté Java. 

La capsule est aussi très simplifiée, les tests d'existance des méthodes classes etc est aussi géré par COaml-Java.

\ 
\newline
\noindent
$[\![ {\color{blue}class}\ CLASS\ 
 {\color{blue}extends}\  E \ 
 {\color{blue}implements}\  I1,I2... \{$

 $ \ \ \ attr1; attr2; ...;$

  $\ \ \ m1; m2; ...;$

  $\ \ \ init1; init2; ...;$

 $\} ]\!]_{\rho,CB}\longrightarrow$
% 1 type objet t 
% 1 classe encapsulante W de type t
% 1 a n classes (Ci), sous-classes de W (1 par constructeur)
% 1 fonction instanceof pr ce type
% 1 fonction de cast pour ce type
\ 
\newline

\begin{OCaml}

(** type 'a java_instance*)
"type _jni_jCLASS = PACK'CLASS java_instance;;"

(** classe encapsulante *)
"class type jCLASS =
   object inherit E
   inherits jI1
   inherits jI2 ...
   method _get_jni_jCLASS : _jni_jCLASS
   end"

(* capsule wrapper *)
"class _capsule_jCLASS = 
  fun (jni_ref : _jni_jCLASS) ->
     let _ =
        if Java.is_null jni_ref
        then raise (Null_object "mypack/Point")
        else ()
     in

    object (self)
     (* method _get_jni_jCLASS = jni_ref
      method _get_jni_jE = jni_ref
      method _get_jni_jI1 = jni_ref
      method _get_jni_jI2 = jni_ref*)
      inherit JniHierarchy.top jni_ref
    end"

(* downcast utilisateur *)
"let jCLASS_of_top (o : TOP) : jCLASS =
    new _capsule_jCLASS (__jni_jCLASS_of_jni_obj o#_get_jniobj)"
(* instance_of *)
"let _instance_of_jCLASS =
    in fun (o : TOP) -> Jni.is_instance_of o#_get_jniobj clazz"


\end{OCaml}
\ 
\newline
\noindent
\textbf{ methodes } 

Module Java, module Javastring et des types array
\url{http://ocamljava.x9c.fr/preview/javalib/index.html}

Tableau représentant les équivalents en OCaml des types Java manipulés.
La troisième colonne représente les types manipulés par les programmes OCaml.
Le problème est donc de convertir du deuxième au troisième type pour la manipulation côté OCaml et du troisième au second lors d'un appel à une fonction du module Java (un appel, un constructeur ou autre).
%% The following table shows how Java primitive types are mapped to OCaml predefined types. 



\begin{tabular}{|c|l|l|l|}
 \hline
TYPE IDL &type Java & type OCaml  pour OCaml-Java & type OCaml \\
& (java\_type) & (oj\_type t) & (ml\_type t) \\
 \hline
\emph{void} & void & unit & unit\\
\emph{boolean} &boolean & bool & bool\\
\emph{byte} & byte & int & int \\
\emph{char} &char & int & char\\
\emph{double} & double & float & float\\
\emph{float} & float & float & float\\
\emph{int} & int & int32 & int\\
\emph{long} & long & int64 & int\\
\emph{short} & short & int & int\\
\emph{string} & java.lang.String & java'lang'String java\_instance & string\\
\emph{pack/Obj} & pack.Obj & pack'Obj java\_instance & jObj\\
 \hline
\end{tabular}
\
\newline
Tableau associant pour chaque types de l'IDL les
fonctions utiles aux schémas de compilation manipulant ceux-ci, comme explicité ci-dessus. 


\begin{tabular}{|c|l|l|l|}
  \hline

  TYPE IDL & to\_oj\_Type ARGi & to\_ml\_type ARGi & fcast\\

  \hline
  \emph{void} &   &  & \\

  \emph{boolean} &  &  &\_pi \\

 \emph{byte} & &  &\_pi  \\

 \emph{cha}r & TODO  & TODO & \_pi \\

  \emph{short} & &  & \\

 \emph{int} & Int32.of\_int\ & Int32.to\_int & \_pi\\

 \emph{long} & Int64.of\_int & Int64.to\_int & \_pi\\

 \emph{float} & & &\_pi \\

  \emph{doubl}e & & & \_pi\\

 \emph{string} & JavaString.of\_string & JavaString.to\_string & \_pi\\

 \emph{pack/Obj} & \_pi\#\_get\_jni\_jObj & TODO & (\_pi: jObj)\\

  \hline
\end{tabular}
\ 
\newline
\ 
\newline
\noindent
$[\![ TYPE METH (ARG1, ARG2, ...)]\!]_{}$$\longrightarrow$

\begin{OCaml}
...
(* type class *)
"class type jCLASS =
   ...
   method METH : ARG1 -> ARG2 -> ... -> RTYPE
   ... "
(* capsule *)
"class _capsule_jCLASS =
   object (self)      
      method METH =
         fun "(fcast ARG0) (fcast ARG1) ..." ->
           let _p1 = "(to_oj_type ARG1)" _p1 in
           let _p0 = "(to_oj_type ARG0)" _p0
           in"
             (to_ml_type RTYPE)
             "Java.call \"PACK.CLASS.METH("(javaType ARG1),(javaType ARG2),...):(javaType RTYPE)"\" jni_ref _p0 _p1 ..."

\end{OCaml}
\ 
\newline
\noindent
\textbf{ inits }
\newline
\noindent
$[\![$[$ {\color{blue}name}\ INIT $]${\color{blue} <init>} (ARG0,
    ARG1, ...)]\!]_{}$$\longrightarrow$
% 

\begin{OCaml}
"class INIT _p0 _p1 ... =
  let _p1 = "(to_oj_type ARG1)"  in
  let _p0 = "(to_oj_type ARG2)" in
  let java_obj = Java.make \"PACK.CLASS("(javaType
           ARG1),(javaType ARG2),...")\" _p0 _p1
  in 
  object (self) 
     inherit _capsule_jCLASS java_obj 
  end;;"

\end{OCaml}

\ 
\newline
\noindent
\textbf{ attributs }
\newline
\noindent
\ 
$[\![ TYPE\ ATTR; ]\!]_{}$$\longrightarrow$

\begin{OCaml}
...
(* type class *)
"class type jCLASS =
  ...
   method set_ATTR : (j)TYPE -> unit
   method get_ATTR : unit -> (j)TYPE
   ... "
(* capsule *)
"class _capsule_jCLASS =
   ...
   fun (jni_ref : _jni_jCLASS) -> 
     object (self)
     ...
        method set_ATTR =
           fun "(fcast TYPE)" ->
              let _p = "(to_oj_type TYPE)"
              in Jni.set_object_field jni_ref __fid_ATTR _p
        method get_ATTR =
        fun () ->
           "(to_ml_type TYPE)"
           (new _capsule_jCLASS (Jni.get_object_field jni_ref __fid_ATTR) :
           jCLASS)
        ...
   "

  method set_y =
    fun _p ->
      let _p = Int32.of_int _p
      in Java.set "mypack.Point.y:int" jni_ref _p
  method get_y =
    fun () -> Int32.to_int (Java.get "mypack.Point.y:int" jni_ref)
\end{OCaml}











%%%%%%%%%%%%%%%%%%%%%%%%%%%%   ANNEXE   %%%%%%%%%%%%%%%%%%%%%%%%%%%%%%


\newpage
\section*{Annexe}


\subsection*{BNF}
\begin{idl}
(*\class*)

file ::= (*\package*) <(*\package*)>*
  	| decl <decl>*
 
(*\package*) ::= package qname ; decl <decl>*

decl ::= (*\class*)
  	|(*\interface*)
 
(*\class*) ::= <[attributes]> <abstract> class (*\name*)
  	  < extends qname >
  	  < implements qname <, qname>* >
  	  { <class_elt ;>* }
class_elt ::= <[ attributes ]> <static> <final> type (*\name*)
            | <[ attributes ]> <static> <abstract> type (*\name*) (<args>)
            | [ attributes ] <init> (<args>)
 
(*\interface*) ::= <[ attributes ]> interface (*\name*)
  	       < extends qname <, qname>* >
  	      { <interface_elt;>* }
interface_elt ::= 
     <[ attributes ]> type (*\name*)
   | <[ attributes ]> type (*\name*) (<args>)
 
args ::= arg <, arg>*
arg ::= <[ attributes ]> type <(*\name*)>
 
attributes ::= 	attribute <, attribute>*
attribute ::= name (*\ident*)
  	    | callback
  	    | array
 
type ::= basetype
       | object
       | basetype [ ]
basetype ::= void
           | boolean
           | byte
           | char
           | short
           | int
           | long
           | float
           | double
           | string
object := qname
qname ::= (*\name*)<.(*\name*)>*
(*\name*) ::= (*\ident*)
\end{idl}
\newpage
\subsection*{Module Idl}
\begin{multicols}{2}
\begin{OCaml}
(**  module Idl  *)

type ident = {
    id_location: Loc.t;
    id_desc: string 
  }
type qident = {
    qid_location: Loc.t;
    qid_package: string list;
    qid_name: ident;
  }
type type_desc = 
    Ivoid  
  | Iboolean
  | Ibyte
  | Ishort
  | Icamlint
  | Iint
  | Ilong
  | Ifloat
  | Idouble
  | Ichar
  | Istring
  | Itop
  | Iarray of typ
  | Iobject of qident
and typ = {
    t_location: Loc.t;
    t_desc: type_desc;
  }
type modifier_desc = 
  | Ifinal 
  | Istatic 
  | Iabstract
and modifier = {
    mo_location: Loc.t; 
    mo_desc: modifier_desc;
}
type ann_desc =
  | Iname of ident
  | Icallback
  | Icamlarray
and annotation = {
    an_location: Loc.t; 
    an_desc: ann_desc;
}
type arg = {
    arg_location: Loc.t; 
    arg_annot: annotation list;
    arg_type: typ
}   
type init = {
    i_location: Loc.t;
    i_annot: annotation list; 
    i_args: arg list;
}   
type field = {
    f_location: Loc.t;
    f_annot: annotation list; 
    f_modifiers: modifier list;
    f_name: ident;
    f_type: typ
}
type mmethod = { 
    m_location: Loc.t;
    m_annot: annotation list;
    m_modifiers: modifier list;
    m_name: ident;
    m_return_type: typ;
    m_args: arg list
}
type content = 
    | Method of mmethod 
    | Field of field
type def = {
    d_location: Loc.t;
    d_super: qident option;
    d_implements: qident list;
    d_annot: annotation list;
    d_interface: bool;
    d_modifiers: modifier list;
    d_name: ident;
    d_inits: init list;
    d_contents: content list;
}
type package = {
    p_name: string list;
    p_defs: def list;
}   
type file = package list
 
\end{OCaml}
\end{multicols}
\newpage
\subsection*{Module CIdl}
\begin{OCaml}
(**  module CIdl  *)
type typ =
  | Cvoid
  | Cboolean (** boolean -> bool *)
  | Cchar (** char -> char *)
  | Cbyte (** byte -> int *)
  | Cshort (** short -> int *)
  | Ccamlint (** int -> int<31> *)
  | Cint (** int -> int32 *)
  | Clong (** long -> int64 *)
  | Cfloat (** float -> float *)
  | Cdouble (** double -> float *)
  | Ccallback of Ident.clazz
  | Cobject of object_type (** object -> ... *)
and object_type = 
  | Cname of Ident.clazz (** ... -> object *)
  | Cstring (** ... -> string *)
  | Cjavaarray of typ (** ... -> t jArray *) 
  | Carray of typ (** ... -> t array *) 
  | Ctop

type clazz = {
    cc_abstract: bool;
    cc_callback: bool;
    cc_ident: Ident.clazz;
    cc_extend: clazz option; (* None = top *)
    cc_implements: clazz list;
    cc_all_inherited: clazz list; (* tout jusque top ... (et avec les interfaces) sauf elle-meme. *)
    cc_inits: init list;
    cc_methods: mmethod list; (* methodes + champs *)
    cc_public_methods: mmethod list; (* methodes declarees + celles heritees *)
    cc_static_methods: mmethod list; 
  }
and mmethod_desc = 
  | Cmethod of bool * typ * typ list (* abstract, rtype, args *)
  | Cget of typ
  | Cset of typ
and mmethod = {
    cm_class: Ident.clazz;
    cm_ident: Ident.mmethod; 
    cm_desc: mmethod_desc;
  }         
and init = {
    cmi_ident: Ident.mmethod;
    cmi_class: Ident.clazz;
    cmi_args: typ list;
  }
type file = clazz list
\end{OCaml}

\subsection*{module Ident}
\begin{OCaml}
(* module Ident  *)
(* le type des identifiants de classe de l'IDL *)
type clazz = {
    ic_id: int;
    ic_interface: bool;
    ic_java_package: string list;
    ic_java_name: string;
    ic_ml_name: string;
    ic_ml_name_location: Loc.t;
    ic_ml_name_kind: ml_kind;
  }
type mmethod = {
    im_java_name: string;
    im_ml_id: int; (** entier unique pour une nom ml *)
    im_ml_name: string;
    im_ml_name_location:Loc.t;
    im_ml_name_kind: ml_kind;
  }
\end{OCaml}
\ 

\subsection*{idl\_camlgen.make ast}
\ 
\newline
Type jni

\emph{MlClass.make\_jni\_type}
\newline
Class type

\emph{MlClass.make\_class\_type}
\newline
Cast JNI

\emph{MlClass.make\_jniupcast}

\emph{MlClass.make\_jnidowncast}
\newline
Fonction d'allocation

\emph{MlClass.make\_alloc}

\emph{MlClass.make\_alloc\_stub}
\newline
Capsule / souche

\emph{MlClass.make\_wrapper}
\newline
Downcast utilisateur

\emph{MlClass.make\_downcast}

\emph{MlClass.make\_instance\_of}
\newline
Tableaux

\emph{MlClass.make\_array}
\newline
Fonction d'initialisation

\emph{MlClass.make\_fun}
\newline
Classe de construction

\emph{MlClass.make\_class}
\newline
fonctions / methodes static

\emph{MlClass.make\_static}


\end{document}
